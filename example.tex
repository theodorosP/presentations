\documentclass{beamer}
\usetheme{PegasusUCF}
\setbeamertemplate{caption}[numbered]
\hypersetup{
    colorlinks,
    citecolor=blue,
    filecolor=blue,
    linkcolor=blue,
    urlcolor=blue
}
%-------------------------------
\title[UCF Departmet of Physics]{CO2/Bi(111)}
\author{Theodoros E. Panagiotakopoulos}
%>>>>UNCOMMENT 1 FOR YOUR DEPTARTMENT
%\newcommand{\deptname}{0}
%\newcommand{\deptname}{1}%STATISTICS
%\newcommand{\deptname}{2}%MATHEMATICS
%\newcommand{\deptname}{3}%ENGINEERING AND COMPUTER SCIENCE
%\newcommand{\deptname}{4}%ELECTRICAL ENGINEERING AND COMPUTER ENGINEERING
%\newcommand{\deptname}{5}%IST INSTITUTE FOR SIMULATION AND TRAINING
\newcommand{\deptname}{6}%PHYSICS
%\newcommand{\deptname}{7}%MECHANICAL AND AEROSPACE ENGINEERING
%\newcommand{\deptname}{8}%COMPLEX ADAPTIVE SYSTEMS LABORATORY
%\newcommand{\deptname}{9}%CENTER FOR ADVANCED TRANSPORTATION SYSTEMS SIMULATION
%\newcommand{\deptname}{10}%CHEMISTRY
%\newcommand{\deptname}{11}%BIOLOGY
%\newcommand{\deptname}{12}%ECONOMICS
%\newcommand{\deptname}{13}%SOCIOLOGY
%\newcommand{\deptname}{14}%PSYCHOLOGY
%\newcommand{\deptname}{15}%BUSINESS AND ADMINISTRATION
%\newcommand{\deptname}{16}%COLLEGE OF SCIENCES

%<<<<<
\date{\vspace{-8cm}}

%--------------------------------



%%%%%%%%%%%%%%%%
\begin{document}
%%%%%%%%%%%%%%%
\begin{frame}
  \titlepage
\end{frame}

\begin{frame}{Overview}
\tableofcontents
\end{frame}

\section{Motivation}

\begin{frame}{Motivation} 
  \begin{block}{(CO2) emissions}
    \begin{itemize}
    \item The world watches out for carbon dioxide (CO2) emissions. The excess gas is the main cause of global warming due to the greenhouse effect.
    \end{itemize}
  \end{block}
  \begin{center}
    \includegraphics[scale=0.4]{co2.png}
    \footnote{ U.S. NOAA (National Oceanic and Atmospheric Administration)}
  \end{center}
\end{frame}


\section{Computational Details}
\begin{frame}{Computational Details} 
  \begin{block}{Input paremeters}
    \begin{itemize}
    \item DFT, employing the projector-augmented wave (PAW) and the plane-wave  basis set
    \item PBE exchange correlation function
\item Kinetic energy cut-off: 500 eV
\item Electronic convergence threshold: $10^{-6}$ eV
\item Supercell: 6x6x6 Bi(111) slab,
	\begin{enumerate}
		\item CO2 atoms on both sides
		\item Vacuum of 15 
	(For 1 ML Pb coverage: 216 Bi, 2 C, 4 O)
	\end{enumerate}		
\item Kpoint mesh: (1x1x1) for relaxation 
    \end{itemize}
  \end{block}
\end{frame}

\section{Configurations}
\subsection{CO2 configurations}
\begin{frame}{CO2 configurations} 
  \begin{block}{CO2}
    \begin{itemize}
    \item There are two configurations for CO2:
	\begin{enumerate}
	\item Two oxygen atoms up
	\item Two oxygen atoms down
	\end{enumerate}	    
    \end{itemize}
  \end{block}
\begin{columns}[onlytextwidth]
\begin{column}{.45\textwidth}
\begin{figure}
  \includegraphics[scale = 0.4]{conf1.png}
  \caption{First configuration}
\end{figure}
\end{column}
\hfill
\begin{column}{.45\textwidth}
\begin{figure}
  \includegraphics[scale = 0.4]{conf2.png}
  \caption{Second configuration}
\end{figure}
\end{column}
\end{columns}
\begin{center}
Absorb the two configurations of CO2 on Bi(111)
\includegraphics[scale=0.12]{bi_111.png}
\end{center}
\end{frame}



\subsection{CO2/Bi(111) configurations}
\begin{frame}{CO2/Bi(111) configurations}
\begin{columns}[onlytextwidth]
\begin{column}{.45\textwidth}
\begin{figure}
  \includegraphics[scale = 0.1]{first_conf.png}
  \caption{First configuration}
\end{figure}
\end{column}
\hfill
\begin{column}{.45\textwidth}
\begin{figure}
  \includegraphics[scale = 0.1]{second_conf.png}
  \caption{Second configuration}
\end{figure}
\end{column}
\end{columns}
\begin{block}{CO2/Bi(111) absorbtion}
\begin{itemize}
\item Calculations were pefromed for 0, 2 and 4 electrons 
\item It seems that CO2 is not being on Bi(111) 
\end{itemize}
\end{block}
\end{frame}

\begin{frame}{CO2/Bi(111) configurations}
\begin{block}{Why charge?}
\begin{itemize}
\item CO2 is stable molecule and hard to adsorb on anything
\item Adsorption happens only if it forms a v shape configuration
\item This can happen only if CO2 receive electron to its $\pi$ antibonding orbital
\item So charging the surface increase the amount of electron transferring to pi* orbital
\end{itemize}
\end{block}
\end{frame}


\begin{frame}{CO2/Bi(111) configurations}
\begin{block}{CO2}
	\begin{itemize}
	\item CO2 could not be absorbed on the surface
	\item two configurations of COOH used
	\item COOH may be absorbed on Bi(111)\cite{oh2018atomic} so H will then be removed
	\end{itemize}
	\end{block}
	\begin{columns}[onlytextwidth]
\begin{column}{.45\textwidth}
\begin{figure}
  \includegraphics[scale = 0.4]{conf3.png}
  \caption{First configuration}
\end{figure}
\end{column}
\hfill
\begin{column}{.45\textwidth}
\begin{figure}
  \includegraphics[scale = 0.4]{conf4.png}
  \caption{Second configuration}
\end{figure}
\end{column}
\end{columns}
\end{frame}

\subsection{COOH/Bi(111) configurations}
\begin{frame}{COOH/Bi(111) configurations}
\begin{block}{COOH}
\begin{itemize}
\item Calculations were pefromed for 0, 2 and 4 electrons 
\item We were able to absorb COOH on Bi(111)
\end{itemize}
\end{block}
\begin{figure}
  \includegraphics[scale=0.2]{COOH_conf1.png}
  \caption{COOH absorbed on Bi(111)}
\end{figure}

\begin{block}{}
\begin{itemize}
\item Next we remove H and we add cat-ion NH4$^{+}$
\end{itemize}
\end{block}
\end{frame}


\subsection{CO2 + NH4$^{+}$/Bi(111)}
\begin{frame}{CO2 + NH4$^{+}$/Bi(111)}
\begin{block}{CO2 + NH4$^{+}$/Bi(111)}
\begin{itemize}
\item After trying many configurations finally we get the one where CO2 stays bended on Bi(111)
\item Calculations were performed 1, 3 and -1 electrons
\item C was initially constrined
\item after relaxation, constraints were removed and new relaxation calculations performed
\end{itemize}
\end{block}
\end{frame}



\begin{frame}{CO2 + NH4$^{+}$/Bi(111)}
\begin{block}{CO2 + NH4$^{+}$/Bi(111)}
\begin{itemize}
\item CO2 is bended for calculations with 1 e\textsuperscript{-} added since this number gives aproximately the potential of -1 eV used in experiments
\end{itemize}
\end{block}
\begin{figure}
\includegraphics[scale=0.2]{nh4.png}
\caption{CO2 \& NH4$^{+}$/Bi(111) after relaxation}
\end{figure}
\end{frame}






\section{Next Steps}
\begin{frame}{Next Steps}
\begin{block}{Next Steps}
\begin{itemize}
\item Target potential calculations are running for NH4\textsuperscript{+} directly above C atoms
\item NH4\textsuperscript{+} will be removed and see if CO2 stays bended by setting the target potential equal to -1 eV
\item Use H\textsubscript{2}O  instead of NH4\textsuperscript{+} and check if we can use H\textsubscript{2}O instead
\end{itemize}
\end{block}
\end{frame}

\begin{frame}{References}
\bibliographystyle{unsrt}
\bibliography{mybib}
\end{frame}

\end{document}
